\documentclass[9pt]{extarticle}
\twocolumn

\author{Justin R Hill}
\title{Resume}


\usepackage{titlesec}
\usepackage{titling}
\usepackage[margin=4em]{geometry}


\titleformat{\section}
{\huge\bfseries\lowercase}
{}
{0em}
{$\triangleright$ }[\titlerule]

\titleformat{\subsection}
{\bfseries\Large\lowercase}
{}
{0em}
{}

\titleformat{\subsubsection}[runin]
{\bfseries}
{}
{0em}
{}[ $\rightharpoonup$]

\titlespacing{\subsubsection}
{1em}{0.5em}{0.5em}

\renewcommand{\maketitle}{
\begin{center}
\huge
J. R. Hill

|

\Large
hiljusti@so.dang.cool | 206.948.5692
\end{center}
}



\begin{document}
\maketitle


\section{Skills}

A polyglot programmer interested in extreme force multiplication and enabling
the ``best tool for the job'' whether it's new support for a cutting-edge build
tool, or the ancient and forgotten tools of the past.

\subsubsection{Large-Scale Software Maintenance}
For the past couple years I've enjoyed building support for new languages and
toolchains at Amazon, supporting a 20+ year old enterprise codebase that serves
more than 500k software projects and 80k developers, and providing centralized
incident response to critical CVEs.

\subsubsection{Programming Languages}
Experience with a variety of build and packaging tools, and more than 100
programming languages spanning many paradigms (Especially OO, Functional, and
Logic programming). Created and maintained real-world applications in dozens of
languages, mentored students in 28 programming languages, and authored
educational material.


\section{Goals}
I'm currently driven by the concept of "done." I want to continue on a path of
developing high-value software that stands on its own. I don't believe that the
only path to job security is to consider all software projects as a
forever-changing "product" that requires an unending backlog of feature requests
and hefty support burdens over time. Instead I am learning to solve very large,
hard problems (especially when the result is pointless human toil) and move to
the next hard problem.


\section{Charity and Learning}

\subsubsection{Exercism}
Contribute to exercism.org, a learning and mentoring platform for programming
languages. Mentored students in 28 languages, and helped develop and launch
tracks for Fortran and Zig.

\subsubsection{Rail}
Crafting Rail, a virtual machine implemented in Rust for concatenative
programming languages. It currently has two experimental language frontends to
its interpreted mode: The Joy-like $dt$ and the Lispy $stap$.

\subsubsection{Sigi}
Developed Sigi, an organization tool for terminal lovers who hate organizing.
It's a todo-list generator distributed through multiple Linux package managers.


\section{Career Highlights}

\subsection{Amazon, Code Foundations, 2020--Now}

\subsubsection{Community Leader}
Created community groups within Amazon for various technologies like "Gradle
Hackers," a collection of 132 engineers and managers from various business
units across Amazon, AWS, and subsidiaries with a strong core of contributors
and advisors. Created, consulted, or assisted with other communities for
languages and software including Kotlin, Scala, Groovy, Flask, C/C++, Zig,
.NET, Apache Maven, Android, Python, R, Dafny, Elixir, Erlang, Ruby, Haskell,
and Crystal.

\subsubsection{Improved Java Language Experience}
Maintained the Java development experience for Amazon, AWS, and subsidiaries
through large-scale software curation, community management, modernizing
templates, improving build tool integrations, and patching foundational software
used transitively by $>$500,000 software projects. I've directly contributed to
more than 6,000 software projects.

\subsubsection{More Tools in Amazon's Toolbox}
In Amazon's older ``Brazil'' codebase, I codified templates for basic Rust, Go,
Kotlin, and NodeJS usage patterns. I also built support for Zig and examples
of cross-compiling for different Operating Systems and Architectures.

In Amazon's newer ``Peru'' codebase, I codified the Java experience, and built
support for Zig, Apache Maven, Apache Ant/Ivy, CMake, SBT and Clojure. I also
provided the earliest examples of Android and Lambda projects as well as the
earliest examples for building C, Fortran, and X86 64-bit Assembly. (GNU
Assembler and NASM)

\subsubsection{``Builder Tools 102''}
Began a cross-org effort (Learning tech, Tech writers, Builder Tools Product
Management, etc) to supplement and extend newhire training for software
engineers in areas where a 101 course was sorely lacking. This became a 102
course launched in 2023.

\subsubsection{Centralized CVE Mitigation}
I provided large-codebase-compatible patches to critical 3p software.

\subsection{AWS Commerce Platform, 2018--2020}

\subsubsection{Replication Validator}
As part of an organization-wide project to build out the first new region of
AWS's billing infrastructure, I designed and built a validation framework based
on an AWS Kinesis and Step Functions architecture. The framework utilized
Infrastructure As Code and allowed multiple teams to perform big-data parity
checks before, during, and after a data migration to ensure that only targeted
data was migrated, and to detect anomalies like write events during migration.

\subsubsection{Cutover Service}
As part of an organization-wide project to build out the first new region of
AWS's billing infrastructure, I designed and built a low-latency cutover service
and library to categorize live traffic and define valid routes across peer
services in multiple regions.

\subsection{AWS Support, 2014--2017}

\subsubsection{Operational Tools}
TODO: TT Kiosk, Link Alchemizer -- Still not replaced as of 2023!

\subsubsection{Automated Refund Application}
(As a Technical Customer Service Agent) Proposed and implemented automation
via JavaScript to very manual web-based refund tool that saved AWS $>$\$1mm in
2014 operating costs.


\end{document}

